\subsection{Chiffrement asymétrique}

\begin{frame}{Chiffrement asymétrique}
  \begin{block}{Définition : chiffrement asymétrique}
    Le chiffrement asymétrique, également connu sous le nom de cryptographie à clé publique, est une technique de chiffrement qui utilise une paire de clés distinctes pour le processus de chiffrement et de déchiffrement des données.

    Cette paire de clés est composée d'une clé publique et d'une clé privée. Elles sont liées mathématiquement. On peut alors chiffre un message avec la clé publique et déchiffrer le message avec la clé privée.
  \end{block}

  Cela permet de recevoir des messages de manière sécurisée sans échange de clé au préalable.

  \begin{enumerate}
    \item Alice publie sa clé publique (par exemple sur son profil GitHub).
    \item Bob va la contacter et lui envoie un message chiffré avec la clé publique d'Alice
    \item Alice déchiffre le message avec sa clé privée.
  \end{enumerate}
\end{frame}

\begin{frame}[fragile]{Chiffrement asymétrique : exemple}
  On considère un algorithme de chiffrement asymétrique générique :

  \begin{minted}{python}
    def cipher(data: str, key: str) -> str:
    def decipher(data: str, key: str) -> str:

    message = "Hello world"
    private_key, public_key = generateKeys()

    result = cipher(message, public_key)  # "&*#&*#QQ#)"
    clear = decipher(result, private_key) # "Hello world"
  \end{minted}
\end{frame}

\begin{frame}{Chiffrement asymétrique : communication sécurisée}
  \input{figures/asymetric-cipher}
\end{frame}

\begin{frame}{Combinaison asymétrique/symétrique}
  L'utilisation des deux techniques de chiffrement permet de communiquer de manière sécuriser sans avoir à faire un échange de clé au préalable.

  \begin{enumerate}
    \item Bob génère son couple de clés $(P_k,S_k)$ et publie sa clé publique $P_k$
    \item Alice veut communiquer avec Bob. Elle génère une clé de chiffrement symétrique $K$.
    \item Alice chiffre $K$ avec la clé publique de Bob $P_k$ et envoie le message chiffré à Bob.
    \item Bob déchiffre le message d'Alice avec sa clé privée $S_k$ et se retrouve donc en possession de la clé $K$.
    \item Par la suite, Alice et Bob utilise la clé $K$ et un algorithme de chiffrement symétrique pour communiquer.
  \end{enumerate}
\end{frame}

\begin{frame}{Algorithmes asymétriques courants}
  \begin{itemize}
    \item RSA
      \begin{itemize}
        \item Basé sur la factorisation de grands nombres
        \item Utilisé pour le chiffrement et la signature
        \item Tailles de clé : 2048-4096 bits
      \end{itemize}
    \item ECC (Elliptic Curve Cryptography)
      \begin{itemize}
        \item Basé sur les courbes elliptiques
        \item Plus efficace que RSA (clés plus courtes)
        \item Courbes populaires : secp256k1 (Bitcoin), Curve25519
      \end{itemize}
    \item ElGamal
      \begin{itemize}
        \item Basé sur le problème du logarithme discret
        \item Utilisé dans PGP/GPG
      \end{itemize}
  \end{itemize}
\end{frame}

\begin{frame}[fragile]{Exemple pratique avec RSA}
  \begin{columns}
    \begin{column}{0.48\textwidth}
      \begin{minted}[fontsize=\scriptsize]{javascript}
const crypto = require('crypto');

// Génération des clés
const { privateKey, publicKey } = 
  crypto.generateKeyPairSync('rsa', {
    modulusLength: 2048,
    publicKeyEncoding: {
      type: 'spki',
      format: 'pem'
    },
    privateKeyEncoding: {
      type: 'pkcs8',
      format: 'pem'
    }
  });
      \end{minted}
    \end{column}

    \begin{column}{0.48\textwidth}
      \begin{minted}[fontsize=\scriptsize]{javascript}
// Chiffrement
const message = 'Message secret';
const encrypted = crypto.publicEncrypt(
  publicKey,
  Buffer.from(message)
);

// Déchiffrement
const decrypted = crypto.privateDecrypt(
  privateKey,
  encrypted
);

console.log(decrypted.toString());
// Affiche: Message secret
      \end{minted}
    \end{column}
  \end{columns}
\end{frame}

\begin{frame}{Applications pratiques}
  \begin{columns}
    \begin{column}{0.5\textwidth}
      HTTPS/TLS :
      \begin{itemize}
        \item Authentification du serveur
        \item Échange de clé de session
        \item Certificats X.509
      \end{itemize}

      SSH :
      \begin{itemize}
        \item Authentification par clé
        \item Connexion sans mot de passe
        \item Forward secrecy
      \end{itemize}
    \end{column}

    \begin{column}{0.5\textwidth}
      Blockchain :
      \begin{itemize}
        \item Adresses de portefeuille
        \item Signature de transactions
        \item Smart contracts
      \end{itemize}

      Email sécurisé :
      \begin{itemize}
        \item PGP/GPG
        \item S/MIME
        \item ProtonMail
      \end{itemize}
    \end{column}
  \end{columns}
\end{frame}

\begin{frame}{Limites et considérations}
  \begin{itemize}
    \item Performance
      \begin{itemize}
        \item 1000-10000× plus lent que le symétrique
        \item Limité à de petites quantités de données
        \item D'où l'utilisation hybride avec le symétrique
      \end{itemize}
    \item Sécurité
      \begin{itemize}
        \item Vulnérable aux attaques quantiques
        \item Importance de la génération aléatoire
        \item Protection des clés privées critique
      \end{itemize}
    \item Gestion des clés
      \begin{itemize}
        \item Distribution des clés publiques
        \item Révocation de certificats
        \item Durée de vie des clés
      \end{itemize}
  \end{itemize}
\end{frame}

\begin{frame}{Comparaison Symétrique vs Asymétrique}
  \begin{center}
    \begin{tabular}{|l|l|l|}
      \hline
      \textbf{Critère} & \textbf{Symétrique} & \textbf{Asymétrique} \\
      \hline
      Performance & Rapide & Lent \\
      Taille des clés & 128-256 bits & 2048-4096 bits \\
      Échange de clés & Difficile & Facile \\
      Usage principal & Données volumineuses & Échange de clés \\
      Algorithmes & AES, ChaCha20 & RSA, ECC \\
      \hline
    \end{tabular}
  \end{center}

  \begin{itemize}
    \item Les deux types sont complémentaires
    \item Utilisés ensemble dans la plupart des protocoles
    \item Chacun a ses forces et faiblesses
  \end{itemize}
\end{frame}