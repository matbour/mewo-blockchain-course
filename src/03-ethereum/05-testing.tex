\subsection{Tester avec Foundry}
\framewithtitle{Tester avec Foundry}

\begin{frame}{Tester un contrat ?}
  Pourquoi ?

  \begin{itemize}
    \item Les contrats sont des programmes exposés aux utilisateurs finaux $\Rightarrow$ un bug peut engendrer \textbf{une perte financière} pour l'utilisateur
    \item Les contrats sont immutables = une fois déployés, il ne peuvent être modifiés
    \item En gros, les contrats doivent être \textbf{parfaits du premier coup}.
  \end{itemize}

  Comment ?

  \begin{enumerate}
    \item Imaginer et créer un scénario (cela revient à définir les variables d'état d'un contrat)
    \item Exécuter la ou les transactions à tester
    \item Vérifier que le nouvel état correspond à celui attendu.
  \end{enumerate}
\end{frame}

\begin{frame}[fragile]{Exemple : Ownable}
  \begin{columns}
    \begin{column}{0.47\textwidth}
      \begin{minted}{solidity}
        contract Ownable {
          address public owner;
    
          modifier onlyOwner() {
            require(msg.sender == owner);
          }
    
          function transferOwnership(address _owner) onlyOwner {
            owner = _owner;
          }
        }      
      \end{minted}
    \end{column}

    \begin{column}{0.47\textwidth}
      Tester la fonction \solidity{transferOwnership}, c'est :

      \begin{enumerate}
        \item \solidity{transferOwnership} renvoie une erreur si \solidity{msg.sender} n'est pas l'\solidity{owner}
        \item \solidity{transferOwnership} ne renvoie pas d'erreur si \solidity{msg.sender} est l'\solidity{owner} et que le nouvel \solidity{owner} est \solidity{msg.sender}
      \end{enumerate}
    \end{column}
  \end{columns}
\end{frame}

\begin{frame}[fragile]{Foundry : créer un test}
  \begin{columns}
    \begin{column}{0.47\textwidth}
      \begin{itemize}
        \item Dans Foundry, les tests s'écrivent en Solidity.
        \item Par convention, on teste le fichier \texttt{Contract.sol} dans \texttt{Contract.\textbf{t}.sol}
        \item Deux méthodes possibles pour placer les tests :
              \begin{itemize}
                \item juste à côté du fichier source
                \item dans le dossier \texttt{test}
              \end{itemize}
      \end{itemize}
    \end{column}
    \hspace{0.03\textwidth}
    \begin{column}{0.47\textwidth}
      \begin{minted}{solidity}
        contract OwnableTest {
          address user1 = makeAddr("user1");
          Ownable ownable;

          function setUp() public {
            Ownable ownable = new Ownable();
            ownable.transferOwnership(user1);
          }

          function test_owner() public {
            assertEq(ownable.owner(), user1);
          }
        }
      \end{minted}
    \end{column}
  \end{columns}
\end{frame}

\begin{frame}[fragile]{Foundry : cheat codes}
  Les cheatcodes permettent d'effectuer des actions normalement impossibles dans la blockchain.
  Voici les principaux :

  \begin{itemize}
    \item \solidity{hoax(address who)} le prochain call sera fait en tant que \solidity{who}
    \item \solidity{hoax(address who, uint256 amount)} le prochain call sera fait en tant que \solidity{who} et sa balance en ether sera de \solidity{amount}
    \item \solidity{deal(address who, uint256 amount)} change la balance de \solidity{who} en \solidity{amount}
    \item \solidity{startHoax(address who)} et \solidity{stopHoax()} permettent de garder l'impersonification sur plusieurs calls
  \end{itemize}
\end{frame}

\begin{frame}[fragile]{Foundry : assertions}
  Les assertions permettent de valider l'exactitude des données / variables.
  Foundry propose notamment les assertions suivantes :

  \begin{itemize}
    \item \solidity{assertEq(x, y)} vérifie que \solidity{x} et \solidity{y} sont égaux
    \item \solidity{assertApproxEqAbs(x, y, maxDelta)} vérifie que \solidity{x} et \solidity{y} sont égaux +/- \solidity{maxDelta}
    \item \solidity{assertTrue(bool cond)} vérifie que \solidity{cond} est égal à \solidity{true}
    \item \solidity{assertFalse(bool cond)} vérifie que \solidity{cond} est égal à \solidity{false}
  \end{itemize}
\end{frame}