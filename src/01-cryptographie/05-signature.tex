\subsection{Signature numérique}

\begin{frame}{Signature numérique : définition}
  \begin{block}{Définition : signature numérique}
    Une signature numérique est un mécanisme cryptographique utilisé pour authentifier l'intégrité et l'origine d'un message, d'un document électronique ou d'un ensemble de données.

    Elle sert à garantir qu'un document n'a pas été altéré depuis sa signature et qu'il provient bien de l'expéditeur prétendu.

    La signature numérique repose sur des algorithmes de cryptographie asymétrique, qui utilisent une paire de clés : une clé privée et une clé publique.

    L'expéditeur utilise sa clé privée pour générer une signature numérique unique basée sur le contenu du document.
    Cette signature est ensuite jointe au document, qui peut être transmis à d'autres parties.
  \end{block}
\end{frame}

\begin{frame}{Signature numérique : exemple}
  \begin{enumerate}
    \item Alice génère sa paire de clé publique/privée $(P_k, S_k)$
    \item Alice publie sa clé publique $P_k$
    \item Alice veut signer le message \textquote{Alice donne 1 BTC à Bob}
          \begin{enumerate}
            \item Alice calcule le hash de \textquote{Alice donne 1 BTC à Bob}
            \item Alice chiffre le hash avec sa clé privée et diffuse le message+le hash signé
          \end{enumerate}
    \item Tout individu peut maintenant vérifier que le message qu'Alice a chiffré est bien le hash du message qu'elle a publié
    \item[$\Rightarrow$] Alice a \textbf{signé} le message \textquote{Alice donne 1 BTC à Bob}
  \end{enumerate}
\end{frame}

\begin{frame}{Algorithmes de signature numérique}
  \begin{itemize}
    \item RSA-PSS
      \begin{itemize}
        \item Version signature de RSA
        \item Padding probabiliste
        \item Largement utilisé
      \end{itemize}
    \item ECDSA
      \begin{itemize}
        \item Basé sur les courbes elliptiques
        \item Utilisé dans Bitcoin
        \item Signatures plus courtes que RSA
      \end{itemize}
    \item EdDSA
      \begin{itemize}
        \item Variante d'ECDSA avec Curve25519
        \item Plus rapide et plus sûr
        \item Utilisé dans SSH et Signal
      \end{itemize}
  \end{itemize}
\end{frame}

\begin{frame}[fragile]{Exemple pratique avec Node.js}
  \begin{columns}
    \begin{column}{0.48\textwidth}
      \begin{minted}[fontsize=\scriptsize]{javascript}
const crypto = require('crypto');

// Génération des clés
const { privateKey, publicKey } = 
  crypto.generateKeyPairSync('ec', {
    namedCurve: 'secp256k1',
    publicKeyEncoding: {
      type: 'spki',
      format: 'pem'
    },
    privateKeyEncoding: {
      type: 'pkcs8',
      format: 'pem'
    }
  });
      \end{minted}
    \end{column}

    \begin{column}{0.48\textwidth}
      \begin{minted}[fontsize=\scriptsize]{javascript}
// Message à signer
const message = "Transaction: Alice -> Bob 1 BTC";

// Signature
const signer = crypto.createSign('SHA256');
signer.update(message);
const signature = signer.sign(privateKey);

// Vérification
const verifier = crypto.createVerify('SHA256');
verifier.update(message);
const isValid = verifier.verify(
  publicKey, 
  signature
);
console.log(isValid ? "Valide!" : "Invalide!");
      \end{minted}
    \end{column}
  \end{columns}
\end{frame}

\begin{frame}{Applications des signatures numériques}
  \begin{columns}
    \begin{column}{0.5\textwidth}
      Documents électroniques :
      \begin{itemize}
        \item Contrats numériques
        \item Factures électroniques
        \item Documents administratifs
      \end{itemize}

      Code et logiciels :
      \begin{itemize}
        \item Signature de code
        \item Mises à jour système
        \item Paquets logiciels
      \end{itemize}
    \end{column}

    \begin{column}{0.5\textwidth}
      Blockchain :
      \begin{itemize}
        \item Transactions
        \item Smart contracts
        \item Gouvernance DAO
      \end{itemize}

      Communication :
      \begin{itemize}
        \item Email (S/MIME, PGP)
        \item Messages instantanés
        \item Certificats TLS
      \end{itemize}
    \end{column}
  \end{columns}
\end{frame}

\begin{frame}{Aspects juridiques}
  \begin{block}{Cadre légal}
    En Europe, le règlement eIDAS définit trois niveaux de signature électronique :
    \begin{itemize}
      \item Simple : basique, peu de valeur légale
      \item Avancée : cryptographiquement sûre
      \item Qualifiée : maximum de valeur légale
    \end{itemize}
  \end{block}

  Exigences pour une signature qualifiée :
  \begin{itemize}
    \item Certificat qualifié
    \item Dispositif sécurisé de création
    \item Horodatage qualifié
    \item Conservation à long terme
  \end{itemize}
\end{frame}

\begin{frame}{Bonnes pratiques}
  \begin{itemize}
    \item Sécurité des clés privées
      \begin{itemize}
        \item Stockage sécurisé (HSM)
        \item Sauvegarde et récupération
        \item Rotation périodique
      \end{itemize}
    \item Vérification
      \begin{itemize}
        \item Validation du certificat
        \item Vérification de révocation
        \item Horodatage
      \end{itemize}
    \item Format et standards
      \begin{itemize}
        \item PAdES pour PDF
        \item XAdES pour XML
        \item CAdES pour données binaires
      \end{itemize}
  \end{itemize}
\end{frame}