\subsection{Chiffrement symétrique}

\begin{frame}{Chiffrement : définition}
  \begin{block}{Définition : chiffrement}
    Le chiffrement est une technique permettant à deux parties d'échanger de manière sécurisée.
    Il existe deux grands types de chiffrements, dits symétrique et asymétrique.

    Les principales caractéritiques du chiffrement sont :

    \begin{itemize}
      \item Confidentialité : protéger contre l'accès non autorisé, seules les personnes ayant la clé puissent déchiffrer et lire le message.
      \item Intégrité : détecter toute modification ou altération des données chiffrées. Si les données chiffrées sont altérées, le déchiffrement donnera un résultat incorrect ou une erreur.
      \item Efficacité : traiter rapidement les données, en particulier lorsqu'il s'agit de volumes importants.
      \item Sécurité : être résistants aux attaques cryptographiques, telles que les attaques par force brute, les attaques de collision, différentielles\dots
    \end{itemize}
  \end{block}
\end{frame}

\begin{frame}{Modèle de canal de communication}
  \begin{figure}
  \resizebox{\columnwidth}{!}{%
    \begin{tikzpicture}
      % Actors
      \node[label=Alice] (alice) at (0,0) {\includegraphics[height=4cm]{img/alice.png}};
      \node[label=Bob] (bob) at (24,0) {\includegraphics[height=4cm]{img/bob.png}};
      \node[label=Eve] (eve) at (12,-4) {\includegraphics[height=4cm]{img/eve.png}};

      % Canal
      \draw (4,0) ellipse (0.35 and 0.5);
      \draw (20,-0.5) arc (-90:90:0.5);
      \draw (4,0.5) -- ++(16,0);
      \draw (4,-0.5) -- ++(16,0);
      \node (label) at (12, 1) {Canal de communication non sécurisé};

      \draw[->] (eve.west) -- ++(-1.5,0) -- node[left] {Écoute} ++(0,3.5);
      \draw[->] (eve.east) -- ++(1.5,0) --  node[right] {Modifie} ++(0,3.5);

    \end{tikzpicture}%
  }

  \caption{Modèle de canal de communication non sécurisé}
\end{figure}
\end{frame}

\begin{frame}{Chiffrement symétrique}
  \begin{block}{Définition : chiffrement symétrique}
    Le chiffrement symétrique est une technique de chiffrement où \textbf{une seule et même clé} est utilisée à la fois pour le \textbf{chiffrement et le déchiffrement} des données.
    Cela signifie que l'émetteur et le destinataire du message doivent partager la même clé secrète pour pouvoir communiquer de manière sécurisée.

    \vspace{1em}

    Exemples d'algorithmes de chiffrement symétriques :

    \begin{itemize}
      \item AES
      \item DES (obslète) et triple DES
    \end{itemize}
  \end{block}
\end{frame}

\begin{frame}{Chiffrement symétrique : modèle}
  \begin{figure}
  \resizebox{\columnwidth}{!}{%
    \begin{tikzpicture}
      % Actors
      \node[label=Alice + clé $K$] (alice) at (0,0) {\includegraphics[height=4cm]{img/alice.png}};
      \node[label=Bob + clé $K$] (bob) at (24,0) {\includegraphics[height=4cm]{img/bob.png}};
      \node[label=Eve sans clé] (eve) at (12,-4) {\includegraphics[height=4cm]{img/eve.png}};

      \node[draw] (alice_clear) at (3,0) {Message};
      \node[draw] (cipher) at (12,0) {\#\&*YU*F};
      \node[draw] (bob_clear) at (21,0) {Message};

      \draw[->] (alice_clear.east) -- (cipher) node[near start,above,sloped]{Chiffre avec $K$};
      \draw[->] (cipher.east) -- (bob_clear) node[near end,above,sloped]{Déchiffre avec $K$};

      % Canal
      \draw (8,0) ellipse (0.35 and 0.5);
      \draw (16,-0.5) arc (-90:90:0.5);
      \draw (8,0.5) -- ++(8,0);
      \draw (8,-0.5) -- ++(8,0);
      \node (label) at (12, 1) {Canal de communication non sécurisé};

      \draw[->] (eve.west) -- ++(-1.5,0) -- node[left] {Écoute} ++(0,3.5);
      \draw[->] (eve.east) -- ++(1.5,0) --  node[right] {Modifie} ++(0,3.5);
    \end{tikzpicture}%
  }

  \caption{Chiffrement symétrique avec une clé $K$}
\end{figure}
\end{frame}

\begin{frame}{Chiffrement symétrique : exercice}
  \begin{block}{Exercice : échange d'information sur canal public}
    \begin{enumerate}
      \item Aller sur \url{https://www.devglan.com/online-tools/aes-encryption-decryption}
      \item Choisir une clé de 32 caractères hexadécimaux (par ex: \texttt{770A8A65DA156D24EE2A093277530142}).
      \item Partager la clé avec un ami.
      \item Chiffrer un message et le publier sur un canal public.
      \item Vérifier que l'ami est capable de déchiffrer le message et personne d'autre.
    \end{enumerate}
  \end{block}
\end{frame}

\begin{frame}{Chiffrement symétrique : résumé}
  \begin{itemize}
    \item Le chiffrement symétrique permet d'échanger des messages secrets.
    \item Une seule clé pour chiffrer et déchiffrer $\Rightarrow$ l'émetteur et le destinataire doivent connaître la clé.
    \item AES est l'algorithme le plus utilisé.
  \end{itemize}

  Problème : comment partager la clé de manière sécurisée ?
\end{frame}