\subsection{Signature numérique}

\begin{frame}{Signature numérique : définition}
  \begin{block}{Définition : signature numérique}
    Une signature numérique est un mécanisme cryptographique utilisé pour authentifier l'intégrité et l'origine d'un message, d'un document électronique ou d'un ensemble de données.

    Elle sert à garantir qu'un document n'a pas été altéré depuis sa signature et qu'il provient bien de l'expéditeur prétendu.

    La signature numérique repose sur des algorithmes de cryptographie asymétrique, qui utilisent une paire de clés : une clé privée et une clé publique.

    L'expéditeur utilise sa clé privée pour générer une signature numérique unique basée sur le contenu du document.
    Cette signature est ensuite jointe au document, qui peut être transmis à d'autres parties.
  \end{block}
\end{frame}

\begin{frame}{Signature numérique : exemple}
  \begin{enumerate}
    \item Alice génère sa paire de clé publique/privée $(P_k, S_k)$
    \item Alice publie sa clé publique $P_k$
    \item Alice veut signer le message \textquote{Alice donne 1 BTC à Bob}
          \begin{enumerate}
            \item Alice calcule le hash de \textquote{Alice donne 1 BTC à Bob}
            \item Alice chiffre le hash avec sa clé privée et diffuse le message+le hash signé
          \end{enumerate}
    \item Tout individu peut maintenant vérifier que le message qu'Alice a chiffré est bien le hash du message qu'elle a publié
    \item[$\Rightarrow$] Alice a \textbf{signé} le message \textquote{Alice donne 1 BTC à Bob}
  \end{enumerate}
\end{frame}