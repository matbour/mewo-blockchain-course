\section{Introduction à la blockchain}
\framewithtitle{Introduction à la blockchain}

\begin{frame}{Sommaire}
    \tableofcontents[currentsection, hideothersubsections]
\end{frame}


\begin{frame}{Objectifs de ce module}
    \begin{enumerate}
        \item Comprendre les enjeux basiques de la blockchain
        \item Développer des smart-contracts de tokens fongibles et non-fongibles
        \item Se sensibiliser à la sécurité de la blockchain
    \end{enumerate}
\end{frame}

\framewithtitle{Blockchain : définitions}

\begin{frame}{Définition générale}
    Blockchain se traduit par \textquote{chaîne de blocs}.
    Il s'agit donc d'un système permettant de stocker et de partager de l'information au travers d'un \textbf{structure de données bien choisie construite à partir de plusieurs blocs} (et c'est tout).

    La majorité des systèmes de blockchain possèdent des caractéristiques supplémentaires qui sont utilisées par abus de langage :

    \begin{enumerate}
        \item Présence d'une cryptomonnaie liée à la blockchain (il existe des blockchains SANS cryptomonnaies)
        \item Décentralisation
        \item Autonome/sans administration centrale
        \item Anonymat/pseudonimat des utilisateurs
    \end{enumerate}
\end{frame}

\begin{frame}{Définitions tierces}
    economie.gouv.fr

    \begin{quote}
        Développée à partir de 2008, c'est, en premier lieu, une technologie de stockage et de transmission d’informations. Cette technologie offre de hauts standards de transparence et de sécurité car elle fonctionne sans organe central de contrôle.

        Plus concrètement, la chaîne de blocs permet à ses utilisateurs - connectés en réseau - de partager des données sans intermédiaire.
    \end{quote}


    Wikipédia

    \begin{quote}
        Une blockchain, ou chaîne de blocs, est une technologie de stockage et de transmission d'informations sans autorité centrale. Techniquement, il s'agit d'une base de données distribuée dont les informations envoyées par les utilisateurs et les liens internes à la base sont vérifiés et groupés à intervalles de temps réguliers en blocs, formant ainsi une chaîne.
    \end{quote}
\end{frame}

\subsection{(Dé)centralisation}
\framewithtitle{(Dé)centralisation}

\begin{frame}{Centralisation}
    Exemples:

    \begin{enumerate}
        \item L'Euro: la banque centrale européenne est souveraine et peut émettre des euros
        \item La force nucléaire en France : contrôlée par l'armée
        \item Twitter : la direction peut décider de retirer des privilèges sans l'approbation des utilisateurs (arrivée d'Elon Musk...)
    \end{enumerate}

    $\Rightarrow$ la centralisation place un privilège/pouvoir entre les mains d'un petit groupe
\end{frame}

\subsection{Cryptographie}
\framewithtitle{Cryptographie}

\subsubsection{Hachage}
\subsubsection{Chiffrement symétrique}
\subsubsection{Chiffrement asymétrique}
\subsubsection{Signatgure digitale}

\subsection{Blockchain}

\subsection{Problème du consensus}
\subsubsection{Proof of work}
\subsubsection{Proof of stack}

\framewithtitle{Blockchain}