
\begin{frame}{Le (sublime) formateur (c'est moi !)}


  \begin{columns}
    \begin{column}{0.5\textwidth}

      \begin{itemize}
        \item Ingénieur spécialisé en Blockchain/Web3
        \item Investisseur dans les cryptomonnaies depuis 2013
        \item Auditeur de smart-contracts depuis 2021
        \item (accessoirement) diplômé des Mines de Saint-Étienne en 2020
        \item Levé \$3M avec \href{https://deepsquare.io}{DeepSquare} sur la blockchain Avalanche
        \item Actuellement chez \href{https://pooky.gg}{Pooky}, un jeu de prédiction de match de foot
      \end{itemize}
    \end{column}

    \begin{column}{0.5\textwidth}  %%<--- here
      \begin{center}
        \includegraphics[width=0.5\textwidth]{img/mathieu.jpg} \\
        Mathieu Bour

        Tél: 06.95.39.72.53 \\
        Mail (Mewo): mathieu.bour@mewo-campus.fr \\
        Mail (pro): mathieu@bour.tech
      \end{center}
    \end{column}
  \end{columns}
\end{frame}


\begin{frame}{Disclaimer}
  \begin{itemize}
    \item DYOR = \textquote{Do Your Own Research} : bien que ce cours soit à jour en mai 2023, je peux m'être trompé. La blockchain n'autorisant pas l'erreur, prenez le temps de faire vos propres recherches.
    \item La blockchain peut permettre de gagner beaucoup d'argent, mais la très grande majorité des investisseurs perdent leur mise. DYOR.
    \item En tant que pro-décentralisation, certaines slides peuvent ne pas être objectives, voire tomber dans la poilitique. DYOR.
    \item Bien que nous évoquerons l'ensemble des types de blockchains, nous utiliserons uniquement les blockchains publiques et décentralisées dans ce cours.
  \end{itemize}
\end{frame}