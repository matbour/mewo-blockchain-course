\framewithtitle{Tokens Standards}

\begin{frame}{Qu'est-ce qu'un token ?}
  \begin{block}{Définition : token}
    Dans le contexte de la blockchain Ethereum, un "token" fait référence à une unité de valeur numérique qui est émise et utilisée sur le réseau Ethereum.
  \end{block}

  \begin{itemize}
    \item La création d'un token sur Ethereum est réalisée en mettant en place un smart contract qui définit les règles et les fonctionnalités spécifiques du token.
    \item Ce smart contract est ensuite déployé sur la blockchain Ethereum, ce qui lui permet d'interagir avec d'autres contrats et d'être utilisé par les utilisateurs du réseau.
    \item Les tokens sur Ethereum peuvent être transférés entre différentes adresses Ethereum, ce qui permet des transactions peer-to-peer.
  \end{itemize}
\end{frame}

\begin{frame}{ERC-20 : besoin de standardisation}
  \begin{itemize}
    \item Le standard ERC-20 a été créé dans le but de faciliter l'émission et l'interopérabilité des tokens sur la blockchain Ethereum.
    \item Avant l'introduction du standard ERC-20, chaque token avait son propre smart contract avec des règles et des interfaces spécifiques, ce qui rendait difficile l'interaction entre les tokens et leur intégration dans des applications tierces.
    \item Le standard ERC-20 définit \textbf{une interface} que doivent suivre les contrats qui l'implémentent.
    \item Grâce à cette norme, les tokens ERC-20 peuvent être facilement créés, gérés, échangés et utilisés par les utilisateurs, les portefeuilles et les plateformes d'échange.
  \end{itemize}
\end{frame}

\begin{frame}{ERC-20 : tokens fongibles}
  Le standard ERC-20 défini les tokens \enquote{fongibles}.

  \begin{block}{Définition : token fongible}
    Un token fongible est un type de token dans lequel chaque unité est interchangeable avec une autre unité du même type.
    Cela signifie que chaque token fongible est considéré comme équivalent et peut être remplacé par un autre token identique, sans qu'il y ait de distinction ou de différence entre eux.
  \end{block}

  \begin{itemize}
    \item C'est la perception \enquote{naturelle} de la monnaie : un billet de 5 euros a exactement la même valeur qu'un autre billet de 5 euros.
    \item Chaque unité d'un token fongible ERC-20 est identique en termes de valeur, de fonctionnalité et de propriétés.
  \end{itemize}
\end{frame}

\begin{frame}[fragile]{ERC-20 : interface}
  \begin{minted}{solidity}
    interface IERC20 {
      function name() public view returns (string)
      function symbol() public view returns (string)
      function decimals() public view returns (uint8)
      function totalSupply() public view returns (uint256)
      function balanceOf(address _owner) public view returns (uint256 balance)
      function transfer(address _to, uint256 _value) public returns (bool success)
      function transferFrom(address _from, address _to, uint256 _value) public returns (bool success)
      function approve(address _spender, uint256 _value) public returns (bool success)
      function allowance(address _owner, address _spender) public view returns (uint256 remaining)

      event Transfer(address indexed _from, address indexed _to, uint256 _value)
      event Approval(address indexed _owner, address indexed _spender, uint256 _value)
    }
  \end{minted}
\end{frame}

\begin{frame}[fragile]{ERC-20 : metadata}
  Le nom et le symbole d'un token ERC-20 sont encodés dans le contrat avec les fonctions \solidity{name} et \solidity{symbol}.

  \begin{minted}{solidity}
    contract ERC20 {
      function name() public view returns (string) {
        return "MyToken";
      }

      function symbol() public view returns (string) {
        return "MYT";
      }
    }
  \end{minted}
\end{frame}

\begin{frame}[fragile]{ERC-20 : decimals}
  Solidity ne supporte pas les nombres flottants donc il n'est a priori pas possible d'échanger 0.5 MYT. La technique consiste à considérer que :

  $$1\text{MYT} = 1\underbrace{000000000000000000}_\text{decimals = 18}$$

  \begin{minted}{solidity}
    contract ERC20 {
      function decimals() public view returns (uint8) {
        return 18;
      }
    }
  \end{minted}
\end{frame}

\begin{frame}[fragile]{ERC-20 : \texttt{transfer} de tokens}
  \begin{itemize}
    \item Le nombre de tokens détenus par une adresse s'obtient avec la fonction \solidity{balanceOf}.
    \item Une adresse peut envoyer des tokens à une autre adresse avec la fonction \solidity{function transfer(address to, uint256 amount)}. Contraintes :\begin{itemize}
            \item \solidity{to} ne peut pas être l'adresse zéro
            \item \solidity{msg.sender} doit posséder au moins \solidity{amount} tokens
          \end{itemize}
  \end{itemize}
\end{frame}

\begin{frame}[fragile]{ERC-20 : \texttt{transferFrom}}
  Intégrer les tokens ERC-20 avec d'autres smart contracts nécessite de pouvoir transférer au nom de l'utilisateur.

  Cela permet notamment d'intégrer des fonctionnalités du type \enquote{envoyez des tokens et pourrez faire la chose X}.

  Exemple : envoyez 1 USDC et receivez 5 MYT.

  \begin{minted}{solidity}
    contract Sale {
      MyToken myt;
      ERC20 usdc;

      function buy(uint256 amount) public {
        usdc.transferFrom(msg.sender, address(this), amount);
        myt.mint(address(this), amount);
      }
    }
  \end{minted}
\end{frame}

\begin{frame}[fragile]{ERC-20 : \texttt{allowance}}
  Problème : il faut protéger la fonction \solidity{function transferFrom}

  \begin{block}{Définition : allowance}
    Dans le contexte du standard ERC-20, \textbf{allowance} fait référence à une fonctionnalité qui permet à une adresse Ethereum spécifique d'accéder à un certain nombre de tokens détenus par une autre adresse Ethereum et de les transférer ultérieurement.

    L'allowance est souvent utilisée dans le cadre des transferts de tokens ERC-20 à des tiers de confiance, tels que des contrats ou des dApps (applications décentralisées).
    Elle permet à un propriétaire de tokens de définir une autorisation spécifique pour un dépensier donné.
    Cette autorisation est enregistrée dans le smart contract du token ERC-20.
  \end{block}
\end{frame}

\begin{frame}[fragile]{ERC-20 : \texttt{allowance}}
  \begin{columns}
    \begin{column}{0.4\textwidth}
      La fonction \solidity{approve} permet de définir l'\solidity{allowance} d'une adresse Ethereum.

      Reprenons le contrat \texttt{Sale} avec les contrats :

      \begin{itemize}
        \item Bob, notre utilisateur
        \item \texttt{USDC}, un stablecoin
        \item \texttt{MyToken}, notre token ERC20
        \item \texttt{Sale}, qui vend des MYT contre des USDC
      \end{itemize}

      Nous allons analyser l'utilisation de l'allowance par Bob.
    \end{column}

    \begin{column}{0.57\textwidth}
      Situation initiale (\solidity{msg.sender = address(Bob)}):

      \begin{itemize}
        \item \solidity{USDC.balanceOf(address(Bob)) = 10000}
        \item \solidity{USDC.allowance(address(Sale)) = 0}
        \item \solidity{MyToken.balanceOf(address(Bob)) = 0}
      \end{itemize}

      Bob envoie \solidity{USDC.approve(address(Sale), 2000)}
      \begin{itemize}
        \item \solidity{USDC.allowance(address(Sale)) = 2000}
      \end{itemize}

      Bob envoie \solidity{Sale.buy(1000)}
      \begin{itemize}
        \item \solidity{USDC.balanceOf(address(Bob)) = 9000}
        \item \solidity{USDC.allowance(address(Sale)) = 1000}
        \item \solidity{MyToken.balanceOf(address(Bob)) = 5000}
      \end{itemize}
    \end{column}
  \end{columns}
\end{frame}

\begin{frame}[fragile]{ERC-20 : \texttt{increaseAllowance} / \texttt{decreaseAllowance}}
  En plus de la fonction \solidity{approve} certaines implémentations exposent des fonction non-standard permettant de contrôller plus finement l'\solidity{allowance}.,
  \begin{itemize}
    \item \solidity{increaseAllowance} qui correspond à l'opération \solidity{allowance[spender] += amount}
    \item \solidity{decreaseAllowance} qui correspond à l'opération \solidity{allowance[spender] -= amount}
  \end{itemize}

  Ces deux fonctions sont notamment présentent dans l'implémentation d'OpenZeppelin.
\end{frame}

\begin{frame}[fragile]{ERC-20 : avec OpenZeppelin}
  \begin{columns}
    \begin{column}{0.47\textwidth}
      \begin{center}

        \resizebox{0.5\textwidth}{!}{
          \includegraphics{img/openzeppelin.png}
        }
      \end{center}

      OpenZeppelin est une bibliothèque de smart-contracts contenant notamment une implémentation du standard ERC-20.

      \begin{itemize}
        \item GitHub : \href{https://github.com/OpenZeppelin/openzeppelin-contracts}{OpenZeppelin/openzeppelin-contracts}
        \item Installer : \mintinline{bash}{forge install openzeppelin/openzeppelin-contracts}
      \end{itemize}
    \end{column}

    \hspace{0.03\textwidth}

    \begin{column}{0.47\textwidth}
      \begin{minted}{solidity}
        import { ERC20 } from "openzeppelin";
  
        contract MyToken is ERC20 {
          constructor() ERC20("MyToken", "MYT") {
            _mint(msg.sender, 1000000000e18);
          }
        }
      \end{minted}

      $\Rightarrow$ c'est tout ! À vous de décider des fonctionnalités de votre token ERC-20 !
    \end{column}
  \end{columns}
\end{frame}

\begin{frame}{ERC-721}
  Le standard ERC-721 défini les tokens \enquote{non-fongibles}, appelées \enquote{Non-Fungible Tokens}.

  \begin{block}{Définition : token non fongible (NFT)}
    Un token non fongible (NFT) est un type de token unique et indivisible sur une blockchain. Chaque NFT est distinct et possède des caractéristiques qui le rendent unique.

    \begin{itemize}
      \item Les NFTs tirent leur valeur de leur unicité, de leur rareté et de leur authenticité, ce qui les rend particulièrement adaptés à la représentation d'actifs numériques uniques et à leur propriété vérifiable sur la blockchain.
      \item Chaque NFT possède un identifiant unique qui le distingue des autres tokens, et ses propriétés et sa provenance sont enregistrées dans le registre immuable de la blockchain.
    \end{itemize}
  \end{block}

  Exemple hors blockchain : certaines anciennes pièces de monnaies valent plus que leur valeur faciale (numismatique).
\end{frame}

\begin{frame}[fragile]{ERC-721 interface}
  \begin{minted}{solidity}
    interface ERC721 {
      function balanceOf(address _owner) external view returns (uint256);
      function ownerOf(uint256 _tokenId) external view returns (address);
      function safeTransferFrom(address _from, address _to, uint256 _tokenId, bytes data) external payable;
      function safeTransferFrom(address _from, address _to, uint256 _tokenId) external payable;
      function transferFrom(address _from, address _to, uint256 _tokenId) external payable;
      function approve(address _approved, uint256 _tokenId) external payable;
      function setApprovalForAll(address _operator, bool _approved) external;
      function getApproved(uint256 _tokenId) external view returns (address);
      function isApprovedForAll(address _owner, address _operator) external view returns (bool);
    }
  \end{minted}
\end{frame}

\begin{frame}{ERC-1155}

\end{frame}