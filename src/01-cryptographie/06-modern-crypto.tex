\subsection*{Protocoles cryptographiques modernes}

\begin{frame}{Protocoles de sécurité modernes}
  \begin{itemize}
    \item TLS/SSL
      \begin{itemize}
        \item Sécurisation des communications web (HTTPS)
        \item Négociation de clés
        \item Authentification des serveurs
      \end{itemize}
    \item SSH
      \begin{itemize}
        \item Connexions sécurisées aux serveurs
        \item Tunneling sécurisé
      \end{itemize}
    \item PGP/GPG
      \begin{itemize}
        \item Chiffrement des emails
        \item Signature de documents
      \end{itemize}
  \end{itemize}
\end{frame}

\begin{frame}{Cryptographie post-quantique}
  \begin{block}{Défi}
    Les ordinateurs quantiques pourront casser facilement RSA et d'autres algorithmes basés sur la factorisation et le logarithme discret.
  \end{block}

  Solutions en développement :
  \begin{itemize}
    \item Réseaux euclidiens
    \item Codes correcteurs d'erreurs
    \item Systèmes multivariés
    \item Cryptographie basée sur les isogénies
  \end{itemize}
\end{frame}

\begin{frame}{Zero-Knowledge Proofs}
  \begin{block}{Définition}
    Protocole permettant de prouver la connaissance d'une information sans la révéler.
  \end{block}

  Applications :
  \begin{itemize}
    \item Authentification anonyme
    \item Transactions confidentielles
    \item Vérification de calculs
    \item Smart contracts privés
  \end{itemize}

  Exemple : zk-SNARKs utilisés dans Zcash pour les transactions privées
\end{frame}

\begin{frame}{Homomorphic Encryption}
  \begin{block}{Définition}
    Système de chiffrement permettant d'effectuer des calculs sur des données chiffrées sans les déchiffrer.
  \end{block}

  Types :
  \begin{itemize}
    \item Partiellement homomorphe (PHE)
    \item Quelque peu homomorphe (SHE)
    \item Complètement homomorphe (FHE)
  \end{itemize}

  Applications :
  \begin{itemize}
    \item Cloud computing sécurisé
    \item Vote électronique
    \item Analyse de données privées
  \end{itemize}
\end{frame} 