\framewithtitle{Foundry}

\begin{frame}{Foundry}
  Dans ce cours, nous allons utiliser Foundry, un utilitaire permettant de développer des smart-contracts simplement.
\end{frame}

\begin{frame}[fragile]{Installation de Foundry : Windows}
  Il faut d'abord installer Windows Subsystem Linux avec :
  \begin{minted}{bash}
    $ wsl --install
  \end{minted}

  WSL demandera de redémarrer, puis au redémarrage de choisir un nom d'utilisateur et un mot de passe.
  Pour confirmer la bonne installation de WSL, exécuter :

  \begin{minted}{bash}
    $ wsl --list
  \end{minted}
\end{frame}

\begin{frame}[fragile]{Installation de Foundry}
  Configurer git avec votre nom / email :

  \begin{minted}{bash}
    $ git config --global user.email "you@example.com"
    $ git config --global user.name "Prénom Nom"
  \end{minted}

  Installer Foundryup (installateur de Foundry)
  \begin{minted}{bash}
    $ curl -L https://foundry.paradigm.xyz | bash
  \end{minted}

  Redémarrer le terminal, lancer Foundryup
  \begin{minted}{text}
    $ foundryup
  \end{minted}

  Redémarrer le terminal, lancer Forge
  \begin{minted}{text}
    $ forge
    forge 0.2.0 (31fcf5a 2023-05-19T00:10:33.861185000Z)
  \end{minted}
\end{frame}

\begin{frame}{Architecture d'un projet Foundry}
  \dirtree{%
    .1 /.
    .2 out\DTcomment{Fichiers compilés}.
    .2 lib\DTcomment{Libraries installées}.
    .2 src\DTcomment{Code source des contrats}.
    .2 test\DTcomment{Test des contrats}.
    .2 .gitmodules.
    .2 foundry.toml\DTcomment{Configuration de Foundry}.
  }
\end{frame}